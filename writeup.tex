\documentclass{article}

\begin{document}



\centerline{\sc \large Phase Two Capstone Writeup}
\vspace{.5pc}

\begin{flushleft}
\textbf{Name:} Joshua Abraham
\vspace{.5pc}

\textbf{Date:} 21 SEP 2017
\vspace{.5pc}

\textbf{Current Module:} Phase Two Capstone
\vspace{.5pc}

\textbf{Project Name:} "Mining"
\vspace{.5pc}

\textbf{Project Goals:}
\vspace{.5pc}
\end{flushleft}

This project was quite large and involved many modules, classes, and unit
tests to create a package that implements an Overlord and Drones to mine
minerals from a map. The Overlord and it's drones operate in a simulation
that uses 'ticks' to denote discrete units of time. The simulation begins
with a limited number of 'refined minerals' which are used to spawn Drones,
who scout and mine the map.

\begin{flushleft}
\textbf{Considerations:}
\vspace{.5pc}
\end{flushleft}

\begin{itemize}
	\item[$\bullet$] The project should make use of concepts learned during
	phase two.
	\item[$\bullet$] The 'mining' package must instantiate an Overlord and
	at least two subclasses of Drone.
	\item[$\bullet$] All Zerg units must have health (minimum of 1) and an
	action method that takes a map context as a parameter.
	\item[$\bullet$] All code in the package must follow the PEP8 guidelines.
	\item[$\bullet$] No work should be performed in the 'master' branch.
	\item[$\bullet$] The Overlord class has 1 second to perform it's action
	method, while the Drones have 1 millisecond.
\end{itemize}

\begin{flushleft}
\textbf{Initial Design:}
\vspace{.5pc}
\end{flushleft}

The project is a Python package, composed of several modules and tests:
\begin{itemize}
	\item [$\cdot$] \textit{area.py}: This module contains the Area class
	that is used by drones to represent their view of their map.
	\item [$\cdot$] \textit{dashboard.py}: This module contains the
	Dashboard class that represents all three maps in the simulation.
	\item [$\cdot$] \textit{overlord.py}: This module contains the
	Overlord class that is responsible for creating, deploying, and 
	returning Drones. The Overlord also commands Drones to mine minerals.
	\item [$\cdot$] \textit{drone/}: This directory contains the drone and 
	location modules.
	\item [$\cdot$] \textit{drone.py}: This file contains the Drone, Scout,
	and Miner classes of drone units. 
	\item [$\cdot$] \textit{location.py}: This file contains the Location
	class, which is used by drones to store their current and adjacent 
	tiles.
	\item [$\cdot$] \textit{path.py}: This file contains functions used by
	the Overlord and Drone classes during pathfinding.
	\item [$\cdot$] \textit{zerg.py}: This file contains the abstract class
	Zerg from which Drones and Overlord inherit from.
	\item [$\cdot$] \textit{tests/}: This directory contains unit tests for
	the package.
	\item [$\cdot$] \textit{runtest.sh + runlinter.sh}: These are bash
	scripts to run unit tests and the pep8 linter utility.
\end{itemize}
However, a double carriage-return is read as a paragraph break.

Like this.  But any carriage-returns after the first two will be 
completely ignored; 




\end{document}